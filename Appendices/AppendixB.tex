\chapter{Código fuente firmwware} % Main appendix title
\label{AppendixB} % For referencing this appendix elsewhere, use \ref{AppendixA}

\lstset{
    language=[Sharp]C, % Configura el lenguaje como C#
    basicstyle=\ttfamily\footnotesize, % Fuente monoespaciada y tamaño de letra pequeño
    keywordstyle=\color{blue}, % Color para palabras clave
    commentstyle=\color{green}, % Color para los comentarios
    stringstyle=\color{red}, % Color para cadenas de texto
    numbers=left, % Números de línea a la izquierda
    numberstyle=\tiny\color{gray}, % Estilo de los números de línea
    stepnumber=1, % Incremento de los números de línea
    breaklines=true, % Cortar las líneas largas automáticamente
    frame=single, % Cuadro alrededor del código
    captionpos=b % Posición de los títulos (b: abajo)
}

\section{Nodo Sensor}

\begin{lstlisting}[caption={Ejemplo de código en C\#}]

using Equipos.DS18B20;
using Equipos.SX127X;
using nanoFramework.Device.OneWire;
using nanoFramework.Hardware.Esp32;
using NanoKernel.Ayudantes;
using NanoKernel.Dominio;
using NanoKernel.DTOs;
using NanoKernel.Hilos;
using NanoKernel.Logging;
using NanoKernel.Nodos;
using System;
using System.Device.Adc;
using System.Device.Gpio;
using System.Threading;

namespace NodoMedidor
{
    public class NodoMedidor : NodoBase
    {
        public override TiposNodo tipoNodo => TiposNodo.MedidorLora;

        private const int segundosSleep = 60*15;

        // -----LORA-----
        private LoRaDevice lora;
        private const double FRECUENCIA = 433e6;

        private const int PIN_MISO = 19;
        private const int PIN_MOSI = 23;
        private const int PIN_CLK = 18;
        private const int PIN_NSS = 5;
        private const int PIN_DIO0 = 25;
        private const int PIN_RESET = 14;
        // ----LED----
        private GpioController gpio;
        private GpioPin led;

        private const int PIN_LED_ONBOARD = 2;
        // ----SENSORES----
        private OneWireHost oneWire;
        private Ds18b20 ds18b20;
        private GpioPin vccSensores;
        private AdcController adc;
        private AdcChannel humedadAdc;
        private AdcChannel bateriaAdcSensor;
        private AdcChannel bateriaAdcAp;

        private const int ONE_WIRE_RX = 16;  
        private const int ONE_WIRE_TX = 17;
        private const int ADC_HUMEDAD = 0;
        private const int ADC_BATERIA = 3;
        private const float ERROR_TEMP = -100;
        
        // ----VARS----
        private MedicionesNodoDto dto;

        private readonly byte[] bufferLora = new byte[LoRaDevice.MAX_LORA_PAYLOAD_BYTES];

        public override void Setup()
        {
            // ----LED----
            gpio = new GpioController();
            led = gpio.OpenPin(PIN_LED_ONBOARD, PinMode.Output);
            // Prendemos el led para avisar que estamos configurando
            led.Write(PinValue.High);   

            // ----LORA----
            Hilo.Intentar(() =>
            {
                lora = new LoRaDevice(
                    pinMISO: PIN_MISO,
                    pinMOSI: PIN_MOSI,
                    pinSCK: PIN_CLK,
                    pinNSS: PIN_NSS,
                    pinDIO0: PIN_DIO0,
                    pinReset: PIN_RESET);
                lora.Iniciar(FRECUENCIA);
            }, "Lora", accionException: () => { lora?.Dispose(); });

            // ----SENSORES----
            adc = new AdcController();
            humedadAdc = adc.OpenChannel(ADC_HUMEDAD);
            bateriaAdcSensor = adc.OpenChannel(ADC_BATERIA);

            Configuration.SetPinFunction(ONE_WIRE_RX, DeviceFunction.COM3_RX);
            Configuration.SetPinFunction(ONE_WIRE_TX, DeviceFunction.COM3_TX);

            ds18b20 = new Ds18b20(new OneWireHost(), null, false, TemperatureResolution.VeryHigh);
            ds18b20.IsAlarmSearchCommandEnabled = false;

            Hilo.Intentar(() =>
            {
                if (!ds18b20.Initialize())
                    throw new Exception("No se pudo conectar al sensor de temperatura");
            }, intentos: 3);

            // ----DTO----
            dto = new MedicionesNodoDto() { serial_number = Config.NumeroSerie };

            // Avisamos que terminamos de configurar
            led.Write(PinValue.Low);  
        }

        public override void Loop(ref bool activo)
        {
            try
            {
                dto.AgregarMedicion(MedirBateria(), TiposMediciones.Bateria);
                dto.AgregarMedicion(MedirTemperatura(), TiposMediciones.Temperatura);
                dto.AgregarMedicion(MedirHumedad(), TiposMediciones.Humedad);
                dto.last_updated = DateTime.UtcNow;

                long length = dto.ToBytes(bufferLora);
                lora.Enviar(bufferLora, 0, (int)length);

                Logger.Debug($"Paquete {dto.last_updated.Ticks} enviado, {length} bytes");
                Blink();
            }
            catch (Exception ex)
            {
                Logger.Log(ex);
            }
            finally
            {
                dto.measurements.Clear();

                LimpiarMemoria();

                lora.ModoSleep();

                Logger.Debug($"Sleep por {segundosSleep}seg");

                aySleep.DeepSleepSegundos(segundosSleep);

                //Thread.Sleep((int)(segundosSleep * 1000));
            }
        }

        private float MedirHumedad()
        {
            
           // La matematica del sensor es la siguiente
           // Vsensor = (analogread/1023)*5
           // y = -5,9732x^3 + 63,948x^2 - 232,8x + 308,98 
        
            int analogValue = humedadAdc.ReadValue();

            float vSensor = (analogValue / 4095f * 3.3f);
            double humidityPercentage = (-5.9732 * vSensor * vSensor * vSensor) + (63.948 * vSensor * vSensor) - 232.8 * vSensor + 308.98;

            return analogValue;
        }

        private float MedirTemperatura()
        {
            if (!ds18b20.TryReadTemperature(out var currentTemperature))
            {
                Logger.Error("Temperatura: Error de lectura!");
                return ERROR_TEMP;
            }
            else
            {
                Logger.Debug($"Temperatura: {currentTemperature.DegreesCelsius.ToString("F")}\u00B0C");
                return (float)currentTemperature.DegreesCelsius;
            }
        }

        private float MedirBateria()
        {
            int analogValue = bateriaAdcSensor.ReadValue();
            float vSensor = analogValue / 4095f * 3.3f;

            // Mapeo de cotas con el ADC
            // y = a x + b
            // 0 = a * 2.52 V + b
            // 100 = a * 3.3 V + b
            // y = 128.21 * x - 323.06
            double bateriaPorcentaje = 128.21 * vSensor - 323.06;
            if (bateriaPorcentaje > 100) bateriaPorcentaje = 100;
            if (bateriaPorcentaje < 0) bateriaPorcentaje = 0;

            Logger.Debug($"Bateria: {bateriaPorcentaje}");
            return analogValue;
        }

        private void Blink(int milis = 100)
        {
#if DEBUG
            led.Write(PinValue.High);
            Thread.Sleep(milis);
            led.Write(PinValue.Low);
#endif
        }
    }
}

\end{lstlisting}

\section{Nodo AP}

\begin{lstlisting}[caption={Ejemplo de código en C\#}]

using Equipos.SX127X;
using NanoKernel.Ayudantes;
using NanoKernel.Comunicacion;
using NanoKernel.Dominio;
using NanoKernel.DTOs;
using NanoKernel.Herramientas.Comunicacion;
using NanoKernel.Herramientas.Medidores;
using NanoKernel.Hilos;
using NanoKernel.Logging;
using NanoKernel.Nodos;
using System;
using System.Collections;
using System.Device.Adc;
using System.Device.Gpio;
using System.IO;
using System.Threading;

namespace NodoAP
{
    public class NodoAP : NodoBase
    {
        public override TiposNodo tipoNodo => TiposNodo.AccessPointLora;

        /// ------------------------------
        /// CONFIGURACION ROUTER MOVIL 4G
        private const int tamanioCola = 75; 
        private const int ventanaDesencolamiento = 20; 
        private const int clientTimeoutSeconds = 20;
        private const int intentosEnvioMediciones = 1; 
        private const int milisIntentoEnvioMediciones = 100;
        private const int segundosMedicionNodoAp = 60 * 1;
        /// ------------------------------
        private SmartCompostClient cliente;
        private GpioPin led;
        private LoRaDevice lora;
        private const int PIN_MISO = 19;
        private const int PIN_MOSI = 23;
        private const int PIN_CLK = 18;
        private const int PIN_NSS = 5;
        private const int PIN_DIO0 = 25;
        private const int PIN_RESET = 14;
        private const double FREQ_LORA = 433e6;
        /// ------------------------------
        private const int ADC_CHANNEL_BATERIA = 3;
        private AdcController adcController;
        private AdcChannel adcBateria;
        /// ------------------------------
        private ConcurrentQueue colaMedicionesNodo = new ConcurrentQueue(tamanioCola);
        private ArrayList mensajesDesencolados = new ArrayList();
        private MedicionesApDto payloadEnvioMediciones = new MedicionesApDto();
        /// ------------------------------
        private MedicionesNodoDto medicionNodo;
        private Medidor medidor;
        private int mensajesMedicionAP = 0;
        private const string M_TIRADOS = "tirados";
        private const string M_RECIBIDOS = "encolados";
        private const string M_ENVIADOS = "enviados";
        private const string ERRORES = "errores";

        public override void Setup()
        {
            /// Configuramos el LED
            var gpio = new GpioController();
            led = gpio.OpenPin(2, PinMode.Output);
            /// Prendemos el led para avisar que estamos configurando
            led.Write(PinValue.High);

            /// Conectamos a internet
            Hilo.Intentar(() =>
            {
                if (ayInternet.ConectarsePorWifi(Config.RouterSSID, Config.RouterPassword) == false)
                {
                    throw new Exception("Fallo la conexion al router");
                }

                string ip = ayInternet.ObtenerIp();
                if (ip == "0.0.0.0")
                    throw new Exception("No se pudo asignar la ip");

                /// IP asignada
                Logger.Log($"Ip asignada: {ip}");

            }, $"Wifi");

            /// Vemos si podemos pingear la api
            bool ping = ayInternet.Ping(Config.SmartCompostHost);
            if (ping == false)
                Logger.Log("NO HAY PING AL SERVIDOR");

            /// Cliente
            cliente = new SmartCompostClient(Config.SmartCompostHost, Config.SmartCompostPort, clientTimeoutSeconds);

            /// Configuramos el Lora
            Hilo.Intentar(() =>
                {
                    lora = new LoRaDevice(
                        pinMISO: PIN_MISO,
                        pinMOSI: PIN_MOSI,
                        pinSCK: PIN_CLK,
                        pinNSS: PIN_NSS,
                        pinDIO0: PIN_DIO0,
                        pinReset: PIN_RESET);
                    lora.Iniciar(FREQ_LORA);
                    lora.ModoRecibir();
                },
                "Lora",
                accionException: () =>
                {
                    lora?.Dispose();
                });

            /// Mediciones del Nodo AP
            medicionNodo = new MedicionesNodoDto();
            medicionNodo.serial_number = Config.NumeroSerie;
            medicionNodo.last_updated = DateTime.UtcNow;

            // Avisamos que nos despertamos
            medicionNodo.AgregarMedicion(1, TiposMediciones.Startup);
            Hilo.Intentar(() => cliente.AddNodeMeasurments(Config.NumeroSerie, medicionNodo), intentos: 3);
            medicionNodo.measurements.Clear();

            // Inicializamos el medidor del Nodo AP
            adcController = new AdcController();
            adcBateria = adcController.OpenChannel(ADC_CHANNEL_BATERIA);
            medidor = new Medidor(segundosMedicionNodoAp * 1000);
            medidor.OnMedicionesEnPeriodoCallback += Medidor_OnMedicionesEnPeriodoCallback;
            medidor.Iniciar();

            // Escuchamos cuando terminamos de configurar
            lora.OnReceive += Device_OnReceive;

            /// Avisamos que terminamos de configurar
            led.Write(PinValue.Low);
        }

        #region RECEPCION DE MENSAJES
        private void Device_OnReceive(object sender, SX127XDevice.OnDataReceivedEventArgs e)
        {
            try
            {
                if (e.Data == null)
                    throw new Exception("Mensaje null recibido");

                Logger.Debug($"{e.Data.Length} bytes recibidos");

                byte[] mensaje = ValidarYProcesarMensaje(e.Data);
                if (mensaje == null)
                {
                    e.Data = null;
                    medidor.Contar(M_TIRADOS);
                    return;
                }

                medidor.Contar(M_RECIBIDOS);

                byte[] mensajeDesbordado = (byte[])colaMedicionesNodo.Enqueue(mensaje);
                if (mensajeDesbordado != null)
                {
                    mensajeDesbordado = null;
                    medidor.Contar(M_TIRADOS);
                    Logger.Debug("Cola mediciones desbordada");
                }
            }
            catch (Exception ex)
            {
                medidor.Contar(ERRORES);
                Logger.Log(ex.Message);
            }
        }

        private byte[] ValidarYProcesarMensaje(byte[] data)
        {
            using (MemoryStream ms = new MemoryStream(data))
            using (BinaryReader br = new BinaryReader(ms))
            {
                var tipoPaquete = (TipoPaqueteEnum)br.ReadByte();
                if (tipoPaquete != TipoPaqueteEnum.MedicionNodo)
                {
                    Logger.Error($"Mensaje {tipoPaquete} no soportado");
                    return null;
                }

                MedicionesNodoDto.SetearTimestamp(data, ms, br);
                return data;
            }
        }
        #endregion

        #region ENVIO DE MENSAJES
        public override void Loop(ref bool activo)
        {
            if (colaMedicionesNodo.IsEmpty())
            {
                /// Si no hay mensajes encolados liberamos el thread
                Thread.Sleep(100);
                return;
            }

            try
            {
                int tamanioCola = colaMedicionesNodo.Count();
                for (int i = 0; i < ventanaDesencolamiento && !colaMedicionesNodo.IsEmpty(); i++)
                {
                    var item = (byte[])colaMedicionesNodo.Dequeue();
                    try
                    {
                        payloadEnvioMediciones.AgregarMediciones(MedicionesNodoDto.FromBytes(item));
                        mensajesDesencolados.Add(item);
                    }
                    catch (Exception ex)
                    {
                        Logger.Log(ex);
                    }
                }

                if (mensajesDesencolados.Count == 0)
                {
                    medidor.Contar(ERRORES);
                    Logger.Error("No se pudo deserealizar nada");
                    return;
                }

                Logger.Debug($"{mensajesDesencolados.Count}/{tamanioCola} mensajes desencolados");

                // Envio del payload
                payloadEnvioMediciones.last_updated = DateTime.UtcNow;
                bool payloadEnviado = Hilo.Intentar(
                    () => cliente.AddApMeasurments(Config.NumeroSerie, payloadEnvioMediciones),
                    nombreIntento: "Envio payload AP",
                    milisIntento: milisIntentoEnvioMediciones,
                    intentos: intentosEnvioMediciones);

                if (payloadEnviado)
                {
                    Blink(100);

                    medidor.Contar(M_ENVIADOS, mensajesDesencolados.Count - mensajesMedicionAP);
                    Logger.Log($"{mensajesDesencolados.Count} mensajes enviados");
                }
                else
                {
                    medidor.Contar(ERRORES);

                    /// Si podemos volvemos a meterlo en la cola, sino los tiro para dejar lugar a nuevos mensajes
                    int indiceReencolado = 0;
                    object obj = null;
                    do
                    {
                        obj = colaMedicionesNodo.Enqueue(mensajesDesencolados[indiceReencolado++]);
                    }
                    while (obj == null && indiceReencolado < mensajesDesencolados.Count);

                    medidor.Contar(M_TIRADOS, mensajesDesencolados.Count - indiceReencolado + 1);
                    Logger.Debug($"{indiceReencolado + 1} mensajes reencolados");
                }
            }
            catch (Exception e)
            {
                medidor.Contar(ERRORES);
                Logger.Log(e);
            }
            finally
            {
                // Limpiamos toda la memoria posible
                mensajesMedicionAP = 0;
                payloadEnvioMediciones.nodes_measurements.Clear();
                mensajesDesencolados.Clear();
                LimpiarMemoria();

                Logger.Debug($"Enviados: {medidor.ContadoTotal(M_ENVIADOS)} | Tirados: {medidor.ContadoTotal(M_TIRADOS)} | Encolados: {colaMedicionesNodo.Count()}");
            }
        }
        #endregion

        #region MEDICIONES AP
        private void Medidor_OnMedicionesEnPeriodoCallback(InstanteMedicion resultado)
        {
            try
            {
                medicionNodo.AgregarMedicion(colaMedicionesNodo.Count(), TiposMediciones.TamanioCola);

                var recibidos = resultado.ContadoEnPeriodo(M_RECIBIDOS);
                if (recibidos > 0)
                    medicionNodo.AgregarMedicion(recibidos, TiposMediciones.MensajesRecibidos);

                var tirados = resultado.ContadoEnPeriodo(M_TIRADOS);
                if (tirados > 0)
                    medicionNodo.AgregarMedicion(tirados, TiposMediciones.MensajesTirados);

                var errores = resultado.ContadoEnPeriodo(ERRORES);
                if (errores > 0)
                    medicionNodo.AgregarMedicion(errores, TiposMediciones.Errores);

                var enviados = resultado.ContadoEnPeriodo(M_ENVIADOS);
                if (enviados > 0)
                    medicionNodo.AgregarMedicion(enviados, TiposMediciones.MensajesEnviados);

                var bateria = MedirBateria();
                if (bateria > 0)
                    medicionNodo.AgregarMedicion(bateria, TiposMediciones.Bateria);

                medicionNodo.last_updated = DateTime.UtcNow;

                colaMedicionesNodo.Enqueue(medicionNodo.ToBytes());
                mensajesMedicionAP++;
                Logger.Debug("Encolando mediciones del Nodo AP");
            }
            catch (Exception ex)
            {
                medidor.Contar(ERRORES);
                Logger.Log(ex);
            }
            finally
            {
                medicionNodo.measurements.Clear();
            }
        }

        private float MedirBateria()
        {
            int analogValue = adcBateria.ReadValue();
            float vSensor = analogValue / 4095f * 3.3f;

            // Cuenta de la bateria, mapeando las cotas con el ADC
            // y = a x + b
            // 0 = a * 2.52 V + b
            // 100 = a * 3.3 V + b
            // y = 128.21 * x - 323.06
            double bateriaPorcentaje = 128.21 * vSensor - 323.06;
            if (bateriaPorcentaje > 100) bateriaPorcentaje = 100;
            if (bateriaPorcentaje < 0) bateriaPorcentaje = 0;

            return analogValue;
        }

        #endregion

        private void Blink(int time)
        {
            led.Write(PinValue.High);
            Thread.Sleep(time);
            led.Write(PinValue.Low);
        }
    }
}

\end{lstlisting}