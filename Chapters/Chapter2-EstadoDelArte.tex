\chapter{Estado Del Arte} % Main chapter title

\label{Chapter2} % For referencing the chapter elsewhere, use \ref{Chapter1} 
\label{EstadoArte}


%----------------------------------------------------------------------------------------

%\section{Introducción}

%----------------------------------------------------------------------------------------
\section{Estado del arte del compostaje en Argentina}
\label{sec:EstadoArteArgentina}

En Argentina, el compostaje se encuentra en una etapa inicial en cuanto a la adopción de tecnologías avanzadas para su monitoreo y control. La mayoría de las prácticas actuales son manuales y tradicionales y son muy pocos los organismos, entes o institutos que han adoptado alguna integración tecnológica (esto se debe principalmente por la falta de legislaciones en cuanto al tratamiento de residuos orgánicos).

\subsection{Desarrollos del INTA (Instituto Nacional de Tecnología Agropecuaria)}

El INTA trabajó hace unos años en una iniciativa donde se desarrollo un prototipo de sensores básicos para medir temperatura y humedad en compostajes rurales \footnote{Ver \url{https://intainforma.inta.gob.ar/desarrollan-sensor-para-optimizar-la-produccion-de-compost/}}. Sin embargo estos avances no han sidos implementados a nivel industrial o comercial hasta donde se ha podido investigar.

\subsection{Prácticas actuales}
A nivel industrial e individual, la mayoría de las maniobras relacionadas al compostaje se desarrollan de manera manual. Estas practicas incluyen:
\begin{itemize}
    \item Monitoreo de temperatura y humedad: la temperatura se miden empleando unos termómetros largos que se insertan en la pila de compost. Esta práctica se lleva a cabo varias veces por semana de manera de garantizar que el material en descomposición no alcance temperaturas críticas. Para la humedad se emplea el método del puño, que consiste en apretar una muestra de materia orgánica y observar si libera agua o se desmorona al soltarla.
    \item Volteo y aireación: en la mayoría de los casos, las pilas de compost son aireadas manualmente (por ejemplo mediante el uso de palas) o con maquinarias no especializadas (por ejemplo excavadoras).
\end{itemize}

Tal como se puede observar, las practicas actuales requieren la presencia de operarios en el lugar tanto sea para el monitoreo como para llevar a cabo acciones correctivas. Estas practicar aumentan la carga laboral y reducen la eficiencia del proceso; y exponen a un ambiente nocivo al operario en cuestión.

\section{Estado Del Arte del Compost a nivel global}
\label{sec:EstadoArteGlobal}

A nivel global las técnicas de compostaje han evolucionado significativamente y han surgido integraciones tecnológicas que permitieron optimizar el proceso, sobretodo en países con fuertes políticas de reciclaje y pioneros en innovación.
Se decidió detallar los casos de Estados Unidos y Japón para ejemplificar los avances a nivel global, ya que en otras partes del planeta y en países similares, los desarrollos se encuentran en fases similares.

\subsection{Estados Unidos}
En Estados Unidos existen soluciones privadas que integran IoT y automatizaciones para el monitoreo y procesamiento del compost. Dentro de ellas se destacan empresas como EcoRich \citep{ECORICH}

También destacan proyectos comunitarios, como el programa \textit{Zero Waste} \citep{ZeroWaste}, \textit{Farm Philly} \citep{FarmPhilly} y \textit{Denver Urban Gardens} \citep{DenverUrban}.

\subsection{Japón}
Entre las distintas tecnologías y proyectos comunitarios que Japón ha adoptado se destacan automatizaciones realizadas en colaboración con una empresa canadiense, \textit{Anaconda Systems Limited}\citep{Anaconda}, donde han desarrollado un sistema para compostar residuos orgánicos en cortos periodos de tiempo (aproximadamente 10 días).

A nivel gubernamental, Japón promueve el programa SATREP \textit{Science and Technology Research Partnership for Sustainable Development} \footnote{Ver \url{https://www.jst.go.jp/global/english/}}donde se hace énfasis en el manejo sostenible de residuos orgánicos.

\section{Conclusión}
El compostaje a nivel mundial se encuentra en un punto de desarrollo en el que, aunque existen avances tecnológicos notables, estos no son ni masivos ni altamente sofisticados en su implementación general. Los sistemas automatizados y las soluciones tecnológicas están presentes principalmente en países desarrollados como Estados Unidos, Japón y algunas naciones europeas, pero su adopción a gran escala es limitada. La mayoría de las prácticas siguen siendo manuales o dependen de métodos tradicionales, sobretodo en países en vías de desarrollo.

Se nota también que los proyectos comunitarios juegan un rol fundamental en la actividad del compostaje y que es de vital importancia la concientizacion social y el respaldo estatal.



