\begin{acknowledgements}
%\addchaptertocentry{\acknowledgementname} % Descomentando esta línea se puede agregar los agradecimientos al índice
\vspace{1.5cm}
Ezequiel Altamirano


Este trabajo marca el cierre de una etapa larga, compleja y, en muchas ocasiones, desafiante. Sin embargo, no es solo un logro personal, sino el resultado de un esfuerzo colectivo que se ha sostenido sobre los hombros de muchísimas personas que, de manera directa o indirecta, me acompañaron a lo largo de este camino.

En primer lugar, quiero agradecer a mi familia, que siempre fue mi sostén más fuerte. Cada paso que di estuvo acompañado de su amor, comprensión y paciencia. En los momentos en los que las fuerzas parecían flaquear, ellos siempre estuvieron ahí, para recordarme que no caminaba solo, que este recorrido lo hacíamos juntos. Su apoyo incondicional me dio la energía necesaria para avanzar.

A mis amigos, por ser un espacio de refugio y risas en los momentos de mayor tensión, pero también por sus charlas profundas y sus perspectivas que me ayudaron a ver las cosas desde otro lugar. Ustedes son una parte fundamental de este logro.

No puedo dejar de agradecer al sistema universitario público argentino, un bastión de igualdad y oportunidades. En un país donde las brechas sociales son profundas, la educación pública es un derecho que sigue resistiendo como motor de transformación social. Mi formación fue posible gracias a la universidad pública, gratuita y de calidad, un espacio que no solo forma profesionales, sino que también forma ciudadanos comprometidos con su realidad. Si hoy estoy aquí, es gracias a esa estructura educativa que no discrimina por condición social, sino que apuesta al potencial de cada individuo.

A los docentes que me guiaron a lo largo de mi carrera, quienes no solo me transmitieron conocimientos, sino también valores. A Pablo y Ariel, directores de nuestro trabajo profesional, que no solo nos enseñaron a investigar, sino que nos dieron herramientas para pensar de manera crítica y comprometida con la realidad social que nos rodea. La academia no es una torre de marfil, sino un espacio que debe estar vinculado a las problemáticas reales de nuestra sociedad, y eso lo aprendí también de ustedes.

Un agradecimiento muy especial a mis compañeros de trabajo práctico, Mati y Erik, con quienes compartí tantas horas de esfuerzo, debates y aprendizaje conjunto. Nuestro trabajo fue la mejor muestra de que el conocimiento se construye colectivamente, que las soluciones y los logros siempre son más profundos cuando se alcanzan en equipo. No quiero dejar afuera a quien se nos adelantó en el camino, Andy, que también compartiste casi este recorrido con nosotros.

Quiero agradecer profundamente a mi novia, Sofi, quien en este último tiempo estuvo a mi lado, brindándome su apoyo durante la realización de este trabajo. Toda tu compañía y todo el tiempo de calidad que me brindaste fueron fundamentales para recargar energías y seguir adelante con este proyecto. Aunque no estoy del todo seguro de haberte podido explicar claramente que estábamos haciendo, siempre me preguntabas, me escuchabas y demostrabas interés en lo que veníamos trabajando.

Este camino no estuvo exento de dificultades. Hubo momentos en los que parecía imposible seguir, en los que la incertidumbre, la frustración y el agotamiento se sentían demasiado grandes. Pero nunca lo hice solo. Si algo aprendí en este proceso es que los logros verdaderos no son individuales, sino colectivos.

Vivimos en una sociedad que, muchas veces, promueve la idea del éxito individual, del héroe solitario que consigue todo por mérito propio. Sin embargo, eso no es más que un mito. La realidad es que ningún logro de esta magnitud es posible sin el apoyo, el acompañamiento y la lucha compartida de quienes nos rodean. No existen héroes individuales, existen colectivos, que luchan, que resisten y que avanzan juntos hacia un horizonte común.

Por eso, este trabajo también es un homenaje a la construcción colectiva, a la solidaridad y al compromiso con el otro. Porque en tiempos de tanta fragmentación y competencia, lo que nos salva y nos permite avanzar es el trabajo mancomunado, la colaboración y la conciencia de que, para alcanzar cualquier sueño, es fundamental estar rodeado de personas que compartan esa visión y te sostengan en el proceso.

Hoy, al cerrar esta etapa, quiero reafirmar mi convicción de que el conocimiento, la educación y el trabajo deben estar al servicio de la transformación social, de la justicia y de la igualdad. No tiene sentido estudiar o trabajar solo para el beneficio personal, sino para contribuir a un mundo más justo y equitativo, donde todos tengamos las mismas oportunidades de realizar nuestros sueños.

Gracias a todos los que estuvieron conmigo en este camino. Este logro no es solo mío, es de todos nosotros.




 

\end{acknowledgements}